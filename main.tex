\documentclass[draft,article]{UsydReport}

%% Variable Definitions %%
\newcommand{\reporttitle}{AERO2705 - Space Engineering 1 \\[1Em] Assignment Style Guide}
\newcommand{\authorname}{Prepared by: Julian Guinane}
\newcommand{\authorSID}{}

\title{Title}
\author{Author}
\date{\today}

\begin{document}
\frontmatter
\begin{titlepage}

\newcommand{\HRule}{\rule{\linewidth}{0.5mm}}
% Defines a new command for the horizontal lines, change thickness here

\center % Center everything on the page
 
%----------------------------------------------------------------------------------------
%	HEADING SECTIONS
%----------------------------------------------------------------------------------------

\textsc{\LARGE University of Sydney}\\[1.5cm] % Name of your university/college


%----------------------------------------------------------------------------------------
%	TITLE SECTION
%----------------------------------------------------------------------------------------

\HRule \\[0.4cm]
{ \huge \bfseries \reporttitle}\\[0.4cm] % Title of your document
\HRule \\[1.5cm]
 
%----------------------------------------------------------------------------------------
%	AUTHOR SECTION
%----------------------------------------------------------------------------------------

\begin{minipage}{0.4\textwidth}
\begin{center}\large 
\authorname\\\vspace{1cm}
Access the \href{https://github.com/Julian-Guinane/AERO2705-Report-Template}{GitHub Repo here}.
\authorSID\\\vspace{1cm}


% ADD YOUR SID
\end{center}
\end{minipage}
\vspace{1mm}

% If you don't want a supervisor, uncomment the two lines below and remove the section above
%\Large \emph{Author:}\\
%John \textsc{Smith}\\[3cm] % Your name

%----------------------------------------------------------------------------------------
%	DATE SECTION
%----------------------------------------------------------------------------------------

{\large \today}\\[2cm] % Date, change the \today to a set date if you want to be precise

%----------------------------------------------------------------------------------------
%	LOGO SECTION
%----------------------------------------------------------------------------------------

\includegraphics[width=0.5\textwidth]{Images/USYDlogo.jpg} % Include a department/university logo - this will require the graphicx package
 
%----------------------------------------------------------------------------------------

\vfill % Fill the rest of the page with whitespace
\end{titlepage}
% \maketitle
\newpage
\tableofcontents
% \newpage
% \listoffigures
% \newpage
% \listoftables

\newpage
\mainmatter
% 
\todo{inline, caption={Contents}}{\begin{itemize}
    \item Sections and Numbered Headings
    \item Equations
    \item Figures and Captions
    \item Tables and Captions
    \item References
\end{itemize}}

\section{Report Structure}
When writing up an assignment, always clearly structure your report with numbered headings and subheadings. Some reports use a standard format such as:
\begin{enumerate}
    \item Introduction
    \item Methodology
    \item Results and Discussion
    \item Conclusions
\end{enumerate}
In some cases, it might be better to structure your report with respect to assignment questions:
\begin{enumerate}
    \item Question 1
    \item Question 2
    \item Question 3
\end{enumerate}
It is up to you to decide what is most suitable.

\section{Equations}
Below are some examples of well formatted equations:
\begin{itemize}
    \item Single line equation: \begin{equation}
    c^2 = a^2 + b^2
\end{equation}
    \item Aligned equations: \begin{align}
    \div{\vb{E}} &= \frac{\rho}{\varepsilon_0} \\
    \div{\vb{B}} &= 0 \\
    \curl{\vb{E}} &= -\pdv{\vb{B}}{t}\\
    \curl{\vb{B}} &= \mu_0\qty(\vb{J} + \varepsilon_0 \pdv{\vb{E}}{t})
\end{align}
    \item Sub-equations: \begin{subequations}
\begin{align}
    \oiint_{\partial\Omega} \vb{E} \vdot \dd{\vb{S}} &= \frac{1}{\varepsilon_0} \iiint_\Omega \rho \dd{V} \\
    \oiint_{\partial\Omega} \vb{B} \vdot \dd{\vb{S}} &= 0 \\
    \oint_{\partial \Sigma} \vb{E} \vdot \dd{\vb{l}} &= -\dv{t} \iint_\Sigma \vb{B} \vdot \dd{\vb{S}}
\end{align}
\end{subequations}
    \item Split equations: \begin{equation}
    \begin{split}
        y_1 + y_2 &= A\sin\qty(k x - \omega t) \\
        &+ A\sin\qty(k x + \omega t)
    \end{split}
\end{equation}
    \item Multiple equations in one line (avoid in general): \begin{multicols}{2}
    \noindent
    \begin{equation}
        C_L = \frac{L}{\frac{1}{2}\rho V^2 S}
    \end{equation}
    \begin{equation}
        C_D = \frac{D}{\frac{1}{2}\rho V^2 S}
    \end{equation}
\end{multicols}
\end{itemize}
Key points to remember are:
\begin{itemize}
    \item Equations must always be numbered and referred to in the body of the text.
    \item It is very rare that equations are placed \textit{inline}. For example, $\curl{\vb{B}} = \mu_0\qty(\vb{J} + \varepsilon_0 \pdv{E}{t})$ is poorly formatted and hard to read. It is much better to place the equation in a separate line and refer to it, as in \Cref{eq:DisplayMath}.
    \item While it is usually up to personal preference, be consistent when defining equations as either:
    \begin{itemize}
        \item Defined inside of sentence:
            \begin{equation}
                c^2 = a^2 + b^2
            \end{equation}
        where $c$ is the hypotenuse side-length, and $a$ and $b$ are the other side lengths.
        \item Or defined separately and referred to, as in \Cref{eq:Refer}, where $\vb{E}$ is the electric field, $\rho$ is the charge density, and $\varepsilon_0$ is the permittivity of free space.
        \begin{equation}\label{eq:Refer}
            \div{\vb{E}} = \frac{\rho}{\varepsilon_0}
        \end{equation}
    \end{itemize}
    Whichever style you choose is normally fine, as long as you only use one throughout your report.
    \item Where a variable represents a vector, always denote is as such using bold notation ($\vb{v}$) or arrow notation ($\va{v}$). Again, remain consistent in using one or the other.
\end{itemize}

\section{Figures}


\clearpage
\backmatter{}
\printbibliography[heading=bibintoc]
\listoftodos[Notes]
% Appendix ignored in word count
%%TC:ignore
% \clearpage
\begin{appendices}
\section{Appendix Example}
When writing reports, you may want to include information that is supplementary to the main body of the work. Generally appendices are not included in page counts. However, information in the appendix is also generally not assessed and you should assume markers will not look at the appendix when grading your work.
% \section{Style Guide}
% \subsection{General}
\begin{itemize}
    \item \href{https://www.ctan.org/tex-archive/info/l2tabu/english/}{l2tabu} is a comprehensive list of dos and don'ts and is considered ``required reading'' for \LaTeX{} users. More info can also be found \href{https://faculty.math.illinois.edu/~hildebr/tex/tips.html}{here} for general tips.
    \item Packages may become obsolete or conflict. \href{http://www.macfreek.nl/memory/LaTeX_package_conflicts}{LaTex Package Conflicts} is a good list of these.
    \item New lines can create unwanted space in the compiled document. To make more readable, use \% to comment blank lines before and after commands.
    \item Never break paragraphs using \verb|\\| or \verb|\par|. Add a blank line instead.
    \item Always make sure any compile errors and warnings have been resolved before completing a document.
\end{itemize}
% 
\subsection{Numbers \& Equations}
\begin{itemize}
    \item \LaTeX{} swallows spaces in math environments so add them where you see fit to make code more readable.
    \item The \verb|phyiscs| package defines many handy commands for typesetting equations in math environments. The full documentation can be found \href{https://ctan.org/pkg/physics?lang=en}{here}. Commonly used commands are in \Cref{tab:PhysicsCommands}.
    \begin{table*}
        \centering
        \caption{physics package commands}
        \label{tab:PhysicsCommands}
        \begin{tabular}{lll}
            \verb|\qty()| & $\verb|\qty(\dots)| \longrightarrow \qty(\dots)$ & automatic brace sizing \\
            \verb|\qty{}| & $\verb|\qty{\dots}| \longrightarrow \qty{\dots}$ & \\
            \verb|\qty[]| & $\verb|\qty[\dots]| \longrightarrow \qty[\dots]$ & \\
            \verb|\vectorbold| & $\verb|\vb{a}| \longrightarrow \vb{a}$ & bolded vectors \\
            \verb|\vectorarrow| & $\verb|\va{a}| \longrightarrow \va{a}$ & arrow bolded vectors \\
            \verb|\vectorunit| & $\verb|\vu{a}| \longrightarrow \vu{a}$ & unit bolded vectors \\
            \verb|\gradient| & $\verb|\grad{\Psi}| \longrightarrow \grad{\Psi}$ &  \\
            \verb|\divergence| & $\verb|\div{\Psi}| \longrightarrow \div{\Psi}$ &  \\
            \verb|\curl| & $\verb|\curl{\Psi}| \longrightarrow \curl{\Psi}$ &  \\
            \verb|\differential| & $\verb|\dd{x}|\ \longrightarrow \dd{x}$ & \\
            & $\verb|\dd[2]{x}|\ \longrightarrow \dd[2]{x}$ & \\
             \verb|\derivative| & $\verb|\dv{x}|\ \longrightarrow \dv{x}$ & \\
            & $\verb|\dv{y}{x}|\ \longrightarrow \dv{y}{x}$ & \\
            & $\verb|\dv[2]{y}{x}|\ \longrightarrow \dv[2]{y}{x}$ & \\
            \verb|\partialderivative| & $\verb|\pdv{x}|\ \longrightarrow \pdv{x}$ & \\
            & $\verb|\pdv{y}{x}|\ \longrightarrow \pdv{y}{x}$ & \\
            & $\verb|\pdv[2]{y}{x}|\ \longrightarrow \pdv[2]{y}{x}$ & \\
        \end{tabular}
    \end{table*}
    \begin{itemize}
        \item The standard set of trig functions is redefined in \verb|physics| to provide automatic braces that behave like \verb|\qty()| when used as \verb|\sin(...)|. 
    \end{itemize}
    \item The \verb|siunitx| package is useful for in-text numbers. Numbers should not be entered in text mode unless they are part of a name (e.g. Pixel 2XL). The full documentation can be found \href{https://ctan.org/pkg/siunitx?lang=en}{here}. Commonly used commands are in \Cref{tab:SiunitxCommands}.
    \begin{table*}
        \centering
        \caption{siunitx package commands}
        \label{tab:SiunitxCommands}
        \begin{tabular}{lll}
            \verb|\num{}| & $\verb|\num{123}| \longrightarrow \num{123}$ & consistent in-text numbers \\
            \verb|\ang{}| & $\verb|\ang{45}| \longrightarrow \ang{45}$ & proper degree symbol \\
            \verb|\si{}| & $\verb|\si{\m\per\s}| \longrightarrow \si{\m\per\s}$ & formatted SI units \\
            \verb|\si{}| & $\verb|\SI{123}{\m\per\s}| \longrightarrow \SI{123}{\m\per\s}$ & formatted SI units with number \\
        \end{tabular}
    \end{table*}
    \item Single line equation: \begin{equation}
    c^2 = a^2 + b^2
\end{equation}
    \item Aligned equations: \begin{align}
    \div{\vb{E}} &= \frac{\rho}{\varepsilon_0} \\
    \div{\vb{B}} &= 0 \\
    \curl{\vb{E}} &= -\pdv{\vb{B}}{t}\\
    \curl{\vb{B}} &= \mu_0\qty(\vb{J} + \varepsilon_0 \pdv{\vb{E}}{t})
\end{align}
    \item Sub-equations: \begin{subequations}
\begin{align}
    \oiint_{\partial\Omega} \vb{E} \vdot \dd{\vb{S}} &= \frac{1}{\varepsilon_0} \iiint_\Omega \rho \dd{V} \\
    \oiint_{\partial\Omega} \vb{B} \vdot \dd{\vb{S}} &= 0 \\
    \oint_{\partial \Sigma} \vb{E} \vdot \dd{\vb{l}} &= -\dv{t} \iint_\Sigma \vb{B} \vdot \dd{\vb{S}}
\end{align}
\end{subequations}
    \item Split equations: \begin{equation}
    \begin{split}
        y_1 + y_2 &= A\sin\qty(k x - \omega t) \\
        &+ A\sin\qty(k x + \omega t)
    \end{split}
\end{equation}
    \item Multiple equations in one line (avoid in general): \begin{multicols}{2}
    \noindent
    \begin{equation}
        C_L = \frac{L}{\frac{1}{2}\rho V^2 S}
    \end{equation}
    \begin{equation}
        C_D = \frac{D}{\frac{1}{2}\rho V^2 S}
    \end{equation}
\end{multicols}
\end{itemize}
% 
\subsection{Figures}
\begin{itemize}
    \item Standard figure: \Cref{fig:Standard}. Note that the \verb|width=\columnwidth| for figures placed in the body.
    \begin{figure}
    \centering
    \includegraphics[width=\columnwidth]{example-image-a}
    \caption{Caption}
    \label{fig:Standard}
\end{figure}
    \item Figure across two columns (specific to twocolumn option being used): \Cref{fig:TwoColumn}. Note that the \verb|width=\textwidth| for figures placed across all columns.
    \begin{figure*}
    \centering
    \includegraphics[width=\textwidth]{example-image-a}
    \caption{Caption}
    \label{fig:TwoColumn}
\end{figure*}
    \item Sub-figures: \Cref{fig:Subfigs}. Note that the third uncaptioned figure can be used for a legend or removed.
    \begin{figure}
    \centering
    \begin{subfigure}[t]{0.49\columnwidth}
    \centering
        \includegraphics[width=\linewidth]{example-image-a}
        \caption{Caption}
        \label{fig:SubfigA}
    \end{subfigure}\hfill
    \begin{subfigure}[t]{0.49\columnwidth}
    \centering
        \includegraphics[width=\linewidth]{example-image-b}
        \caption{Caption}
        \label{fig:SubfigB}
    \end{subfigure}
    % Remove below subfigure if separate legend is not necessary
    \vskip \abovecaptionskip 
    \begin{subfigure}[t]{0.49\columnwidth}
    \centering
        \includegraphics[width=\linewidth]{example-image-c}
    \end{subfigure}
    \caption{Caption}
    \label{fig:Subfigs}
\end{figure}
    \item Multiple figures in one line: \Cref{fig:Multifig1,fig:Multifig2}.
    \begin{figure}
    \centering
    \begin{minipage}[t]{0.49\columnwidth}
      \centering
        \includegraphics[width=\linewidth]{example-image-a}
        \caption{Caption}
        \label{fig:Multifig1}
    \end{minipage}\hfill
    \begin{minipage}[t]{0.49\columnwidth}
      \centering
        \includegraphics[width=\linewidth]{example-image-b}
        \caption{Caption}
        \label{fig:Multifig2}
    \end{minipage}%
\end{figure}
\end{itemize}
% 
\subsection{Tables}
\begin{itemize}
    \item Tables should generally adhere to formatting guidelines defined by the \href{https://ctan.org/pkg/booktabs?lang=en}{booktabs} package. The below templates are provided.
    \item For pages with more than one columns, the \verb|table*| environment can be used to span all columns.
    \item Standard table: \Cref{tab:Standard}. Note that \verb|\makecell[<anchor>]{<text\\text>}| can be used to split headers across multiple lines.
    \begin{table}
    \centering
    \caption{Caption}
    \label{tab:Standard}
    \begin{tabular}{lrr} \toprule
         Header 1 & Header 2 & \makecell[cr]{Header\\2} \\ \midrule
         Item 1 & 123456 & 789012 \\
         Item 2 & abcdef & ghijkl \\ \bottomrule
    \end{tabular}
\end{table}
    \item Subheadings with column rules: \Cref{tab:Subheadings}
    \begin{table}
    \centering
    \caption{Caption}
    \label{tab:Subheadings}
    \begin{tabular}{lrr} \toprule
        & \multicolumn{2}{c}{Header 2} \\ \cmidrule{2-3}
        Header 1 & Subheader 1 & Subheader 2 \\ \midrule
        Item 1 & 123456 & 789012 \\
        Item 2 & abcdef & ghijkl \\ \bottomrule
    \end{tabular}
\end{table}
    \item Set Width Table: \Cref{tab:SetWidth}
    \begin{table}
    \centering
    \caption{Caption}
    \label{tab:SetWidth}
    \begin{tabularx}{\columnwidth}{lrX} \toprule
         Header 1 & Header 2 & Header 3 \\ \midrule
         Item 1 & 123456 & This is a long paragraph that needs to be split over more than one line. This is another long paragraph that needs to be split over more than one line.\\
         Item 2 & abcdef &  This is a long paragraph that needs to be split over more than one line. This is another long paragraph that needs to be split over more than one line.\\ \bottomrule
    \end{tabularx}
\end{table}
    \item \verb|longtable| can also be used to create tables that break over more than one page.
\end{itemize}
% 
\subsection{Cross Referencing} 
\begin{itemize}
    \item Always use \verb|\{Cref{<label>}| for cross-referencing rather than \verb|Figure \ref{<label>}|, as the \verb|cleveref| package includes the label.
    \item Clickable URLs can be typeset with \verb|\href{<url>}{<text>}|, (e.g. \href{www.google.com}{Google}).
    \item References are typeset using \verb|biblatex| and can be printed using \verb|\printbibliography|. References are defined in \verb|refs.bib|, where templates are given for a number of document types.
\end{itemize}
% 
\subsection{Misc.}
\begin{itemize}
    \item ToDo items can be left with the \verb|\todo{<note>}| command and will render in the margin as well as in the Notes section if \verb|\listoftodos[Notes]| is present. \todo{This is what a TODO looks like.}
    \item Code can be typeset with the \verb|listings| package using \verb|\lstinputlisting{<path/to/code>}|. Currently syntax highlighting is set for MATLAB.
    \item Sections can be made landscape using the \verb|landscape| environment.
\end{itemize}
\end{appendices}
% 
%%TC:endignore
\end{document}
