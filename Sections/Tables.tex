\section{Tables}
Tables are another good way to clearly format data. Table formatting rules vary between publications, though for university assignments there are some general rules to follow. As in \Cref{tab:Table}, some key points to remember when formatting tables are:
\begin{itemize}
    \item Table captions are always numbered and go \textbf{above} the table.
    \item As with figures, no table should be included unless it is directly referred to in your writing.
    \item Use appropriate significant figures when listing numbers.
    \item Always include units in the column header when the units for a given column are all the same.
    \item Reduce clutter by removing vertical separation lines and any double rules.
    \item Similarly, don't use colours if it isn't absolutely necessary.
    \item Group column headings where possible to clarify relationships between datasets.
\end{itemize}

\begin{table}
    \centering
    \caption{An example of a well formatted table.}
    \label{tab:Table}
    \begin{tabular}{@{}llr@{}} \toprule
    \multicolumn{2}{c}{Item} \\ \cmidrule(r){1-2}
    Animal & Description & Price (\$)\\ \midrule
    Gnat & per gram & 13.65 \\
    & each & 0.01 \\
    Gnu & stuffed & 92.50 \\
    Emu & stuffed & 33.33 \\
    Armadillo & frozen & 8.99 \\ \bottomrule
    \end{tabular}
\end{table}