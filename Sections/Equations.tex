\section{Equations}
Below are some examples of well formatted equations:
\begin{itemize}
    \item Single line equation: \begin{equation}
    c^2 = a^2 + b^2
\end{equation}
    \item Aligned equations:
        \begin{align}
            \div{\vb{E}} &= \frac{\rho}{\varepsilon_0} \\
            \div{\vb{B}} &= 0 \\
            \curl{\vb{E}} &= -\pdv{\vb{B}}{t}\\
            \label{eq:DisplayMath}\curl{\vb{B}} &= \mu_0\qty(\vb{J} + \varepsilon_0 \pdv{\vb{E}}{t})
        \end{align}
    \item Sub-equations: \begin{subequations}
\begin{align}
    \oiint_{\partial\Omega} \vb{E} \vdot \dd{\vb{S}} &= \frac{1}{\varepsilon_0} \iiint_\Omega \rho \dd{V} \\
    \oiint_{\partial\Omega} \vb{B} \vdot \dd{\vb{S}} &= 0 \\
    \oint_{\partial \Sigma} \vb{E} \vdot \dd{\vb{l}} &= -\dv{t} \iint_\Sigma \vb{B} \vdot \dd{\vb{S}}
\end{align}
\end{subequations}
    \item Split equations: \begin{equation}
    \begin{split}
        y_1 + y_2 &= A\sin\qty(k x - \omega t) \\
        &+ A\sin\qty(k x + \omega t)
    \end{split}
\end{equation}
    \item Multiple equations in one line (avoid in general): \begin{multicols}{2}
    \noindent
    \begin{equation}
        C_L = \frac{L}{\frac{1}{2}\rho V^2 S}
    \end{equation}
    \begin{equation}
        C_D = \frac{D}{\frac{1}{2}\rho V^2 S}
    \end{equation}
\end{multicols}
\end{itemize}
Key points to remember are:
\begin{itemize}
    \item Equations must always be numbered and referred to in the body of the text.
    \item It is very rare that equations are placed \textit{inline}. For example, $\curl{\vb{B}} = \mu_0\qty(\vb{J} + \varepsilon_0 \pdv{\vb{E}}{t})$ is poorly formatted and hard to read. It is much better to place the equation in a separate line and refer to it, as in \Cref{eq:DisplayMath}.
    \item While it is usually up to personal preference, be consistent when defining equations as either:
    \begin{itemize}
        \item Defined separately and referred to, as in \Cref{eq:Refer}, where $\vb{E}$ is the electric field, $\rho$ is the charge density, and $\varepsilon_0$ is the permittivity of free space.
        \begin{equation}\label{eq:Refer}
            \div{\vb{E}} = \frac{\rho}{\varepsilon_0}
        \end{equation}
        \item Or defined inside of sentence:
            \begin{equation}
                c^2 = a^2 + b^2
            \end{equation}
        where $c$ is the hypotenuse side-length, and $a$ and $b$ are the other side lengths.
    \end{itemize}
    Whichever style you choose is normally fine, as long as you only use one throughout your report.
    \item Where a variable represents a vector, always denote it as such using bold notation ($\vb{v}$) or arrow notation ($\va{v}$). Again, remain consistent in using one or the other.
\end{itemize}